\documentclass[aspectratio=169]{beamer}

% Tema e cores
\usetheme{Madrid}
\usecolortheme{default}

% Pacotes
\usepackage[utf8]{inputenc}
\usepackage[brazil]{babel}
\usepackage{graphicx}
\usepackage{listings}
\usepackage{xcolor}
\usepackage{tikz}
\usepackage{fontawesome5}

% Configuração de cores
\definecolor{springgreen}{RGB}{109,179,63}
\definecolor{reactblue}{RGB}{97,218,251}
\definecolor{javacolor}{RGB}{237,139,0}

% Informações do documento
\title{LibShow}
\subtitle{Sistema de Gerenciamento de Biblioteca Acadêmica}
\author{[Nomes dos Integrantes]}
\institute{PUC Minas - Ciência da Computação\\Engenharia de Software 2}
\date{Novembro 2024}

% Logo
\logo{\includegraphics[height=0.8cm]{logo.png}} % Adicionar logo se disponível

% Configuração de código
\lstset{
    basicstyle=\ttfamily\footnotesize,
    keywordstyle=\color{blue},
    commentstyle=\color{gray},
    stringstyle=\color{red},
    breaklines=true,
    frame=single
}

\begin{document}

% ============================================================
% SLIDE 1: TÍTULO
% ============================================================
\begin{frame}
\titlepage
\begin{center}
    \faBook\ \faCode\ \faDatabase
\end{center}
\end{frame}

% ============================================================
% SLIDE 2: AGENDA
% ============================================================
\begin{frame}{Agenda}
\tableofcontents
\end{frame}

% ============================================================
% SEÇÃO 1: INTRODUÇÃO
% ============================================================
\section{Introdução e Objetivo}

\begin{frame}{Contexto do Problema}
\begin{block}{Desafios de Bibliotecas Acadêmicas}
Bibliotecas enfrentam dificuldades no gerenciamento manual:
\begin{itemize}
    \item \faExclamationTriangle\ Controle de empréstimos e devoluções
    \item \faExclamationTriangle\ Gestão de acervo e disponibilidade
    \item \faExclamationTriangle\ Cadastro e autenticação de usuários
    \item \faExclamationTriangle\ Geração de relatórios administrativos
    \item \faExclamationTriangle\ Filas de espera para livros indisponíveis
\end{itemize}
\end{block}
\end{frame}

\begin{frame}{Motivação}
\begin{columns}
\column{0.5\textwidth}
\textbf{Problemas Identificados:}
\begin{itemize}
    \item Processos manuais lentos
    \item Erros no controle
    \item Falta de centralização
    \item Dificuldade em análises
\end{itemize}

\column{0.5\textwidth}
\textbf{Benefícios da Solução:}
\begin{itemize}
    \item \faCheckCircle[regular] Reduzir tempo de atendimento
    \item \faCheckCircle[regular] Eliminar erros humanos
    \item \faCheckCircle[regular] Centralizar informações
    \item \faCheckCircle[regular] Facilitar relatórios
\end{itemize}
\end{columns}
\end{frame}

\begin{frame}{Público-Alvo}
\begin{center}
\begin{tikzpicture}
    \node[draw, circle, fill=blue!20, minimum size=2.5cm] at (0,0) 
        {\begin{tabular}{c}\faUserGraduate\\\textbf{Alunos}\end{tabular}};
    \node[draw, circle, fill=green!20, minimum size=2.5cm] at (4,0) 
        {\begin{tabular}{c}\faUserTie\\\textbf{Bibliotecários}\end{tabular}};
    \node[draw, circle, fill=red!20, minimum size=2.5cm] at (8,0) 
        {\begin{tabular}{c}\faUserShield\\\textbf{Administradores}\end{tabular}};
\end{tikzpicture}
\end{center}

\vspace{0.5cm}

\begin{columns}
\column{0.33\textwidth}
\centering
\small
Consultar acervo\\
Fazer reservas\\
Ver histórico

\column{0.33\textwidth}
\centering
\small
Gerenciar empréstimos\\
Cadastrar livros\\
Processar filas

\column{0.33\textwidth}
\centering
\small
Visualizar relatórios\\
Analisar estatísticas\\
Gestão completa
\end{columns}
\end{frame}

% ============================================================
% SEÇÃO 2: REQUISITOS E MODELAGEM
% ============================================================
\section{Requisitos e Modelagem}

\begin{frame}{Principais User Stories}
\begin{block}{US-01: Autenticação de Usuários}
\textbf{Como} usuário do sistema\\
\textbf{Quero} realizar login com minhas credenciais\\
\textbf{Para} acessar funcionalidades de acordo com meu perfil
\end{block}

\begin{block}{US-08: Registrar Empréstimo}
\textbf{Como} bibliotecário\\
\textbf{Quero} registrar um empréstimo de livro\\
\textbf{Para} controlar livros emprestados
\end{block}

\vspace{0.3cm}
\footnotesize
\faInfoCircle\ Total de 17 user stories detalhadas em 5 épicos
\end{frame}

\begin{frame}{Diagrama de Casos de Uso}
\begin{center}
\includegraphics[width=0.9\textwidth]{use_cases.png}
\end{center}
\footnotesize
\textit{Principais casos de uso: Login, Consultar Acervo, Registrar Empréstimo, Fazer Reserva, Ver Relatórios}
\end{frame}

\begin{frame}{Diagrama de Classes - Modelo de Domínio}
\begin{center}
\begin{tikzpicture}[scale=0.7, transform shape]
    % Usuario
    \node[draw, rectangle, text width=3cm] at (0,3) {
        \textbf{Usuario}
        \rule{\textwidth}{0.4pt}
        - id: Long\\
        - nome: String\\
        - email: String\\
        - senha: String\\
        - cpf: String\\
        - tipo: String
    };
    
    % Emprestimo
    \node[draw, rectangle, text width=3.5cm] at (6,3) {
        \textbf{Emprestimo}
        \rule{\textwidth}{0.4pt}
        - id: Long\\
        - dataEmprestimo: Date\\
        - dataDevolucao: Date\\
        - devolvido: Boolean
    };
    
    % Livro
    \node[draw, rectangle, text width=3.5cm] at (6,0) {
        \textbf{Livro}
        \rule{\textwidth}{0.4pt}
        - id: Long\\
        - titulo: String\\
        - autor: String\\
        - isbn: String\\
        - quantidadeTotal: int\\
        - quantidadeDisp: int
    };
    
    % Reserva
    \node[draw, rectangle, text width=3cm] at (0,0) {
        \textbf{Reserva}
        \rule{\textwidth}{0.4pt}
        - id: Long\\
        - dataReserva: Date\\
        - status: String
    };
    
    % Relacionamentos
    \draw[->] (0.5,2.5) -- (5.5,3.3) node[midway,above] {1};
    \draw[->] (6,2.5) -- (6,1.5) node[midway,right] {*};
    \draw[->] (1,0.5) -- (5.5,0.3) node[midway,below] {*};
    \draw[->] (0,2.5) -- (0,0.5) node[midway,left] {1..*};
\end{tikzpicture}
\end{center}
\end{frame}

% ============================================================
% SEÇÃO 3: ARQUITETURA
% ============================================================
\section{Arquitetura do Sistema}

\begin{frame}{Arquitetura em 3 Camadas}
\begin{center}
\begin{tikzpicture}
    % Camada 1: Frontend
    \node[draw, rectangle, fill=reactblue!30, text width=10cm, minimum height=1.5cm] at (0,4) {
        \textbf{CAMADA 1: PRESENTATION}\\
        \small Frontend (React + Vite) - UI Components, State Management, API Services
    };
    
    % Seta 1
    \draw[->, thick] (0,3.2) -- (0,2.8) node[midway,right] {\small HTTP REST + JWT};
    
    % Camada 2: Backend
    \node[draw, rectangle, fill=springgreen!30, text width=10cm, minimum height=2.5cm] at (0,1.5) {
        \textbf{CAMADA 2: BUSINESS LOGIC}\\
        \small Backend (Spring Boot)\\
        \tiny Controllers $\rightarrow$ Services $\rightarrow$ Repositories $\rightarrow$ Domain\\
        \tiny Security (JWT + Spring Security)
    };
    
    % Seta 2
    \draw[->, thick] (0,0.2) -- (0,-0.2) node[midway,right] {\small JPA/Hibernate};
    
    % Camada 3: Database
    \node[draw, rectangle, fill=blue!20, text width=10cm, minimum height=1.2cm] at (0,-1) {
        \textbf{CAMADA 3: DATA PERSISTENCE}\\
        \small Database (H2 / PostgreSQL) - Tables \& Relationships
    };
\end{tikzpicture}
\end{center}
\end{frame}

\begin{frame}{Padrões de Design Utilizados}
\begin{columns}
\column{0.5\textwidth}
\begin{block}{MVC Pattern}
\begin{itemize}
    \item \textbf{Model}: Entidades JPA
    \item \textbf{View}: Frontend React
    \item \textbf{Controller}: REST Controllers
\end{itemize}
\end{block}

\begin{block}{Repository Pattern}
Spring Data JPA para abstração do acesso a dados
\end{block}

\column{0.5\textwidth}
\begin{block}{Dependency Injection}
Spring IoC Container gerencia dependências
\end{block}

\begin{block}{RESTful API}
Endpoints seguindo padrões REST\\
Stateless com JWT
\end{block}
\end{columns}
\end{frame}

\begin{frame}{Justificativa das Escolhas}
\begin{table}
\small
\begin{tabular}{|l|p{6cm}|}
\hline
\textbf{Tecnologia} & \textbf{Justificativa} \\
\hline
Spring Boot & Framework maduro, ecossistema completo, configuração por convenção \\
\hline
React + Vite & Biblioteca moderna, component-based, build rápido \\
\hline
H2 Database & Embedded para desenvolvimento, fácil setup \\
\hline
JWT & Stateless authentication, escalável, padrão moderno \\
\hline
shadcn/ui & Componentes prontos, acessíveis, customizáveis \\
\hline
\end{tabular}
\end{table}
\end{frame}

% ============================================================
% SEÇÃO 4: IMPLEMENTAÇÃO
% ============================================================
\section{Implementação}

\begin{frame}{Stack Tecnológica}
\begin{columns}
\column{0.5\textwidth}
\textbf{\color{javacolor} Backend}
\begin{itemize}
    \item Java 21
    \item Spring Boot 3.3.4
    \item Spring Data JPA
    \item Spring Security
    \item JWT
    \item H2 Database
    \item JUnit 5 + Mockito
    \item Maven
\end{itemize}

\column{0.5\textwidth}
\textbf{\color{reactblue} Frontend}
\begin{itemize}
    \item React 19
    \item Vite 5
    \item Tailwind CSS 4
    \item shadcn/ui
    \item Radix UI
    \item Lucide React
    \item Axios
\end{itemize}
\end{columns}
\end{frame}

\begin{frame}[fragile]{Controller Layer - Exemplo}
\begin{lstlisting}[language=Java]
@RestController
@RequestMapping("/api/livros")
public class LivroController {
    @Autowired
    private LivroService livroService;
    
    @GetMapping
    public List<Livro> getAllLivros() {
        return livroService.findAll();
    }
    
    @PostMapping
    public Livro createLivro(@RequestBody Livro livro) {
        return livroService.save(livro);
    }
    
    @PostMapping("/{id}/decrease/{quantity}")
    public ResponseEntity<Livro> decreaseQuantity(
            @PathVariable Long id, 
            @PathVariable int quantity) {
        livroService.decreaseAvailableQuantity(id, quantity);
        return ResponseEntity.ok(livroService.findById(id).get());
    }
}
\end{lstlisting}
\end{frame}

\begin{frame}[fragile]{Service Layer - Lógica de Negócio}
\begin{lstlisting}[language=Java]
@Service
public class LivroService {
    @Autowired
    private LivroRepository livroRepository;
    
    public void decreaseAvailableQuantity(Long id, int qty) {
        Livro livro = livroRepository.findById(id)
            .orElseThrow(() -> 
                new RuntimeException("Livro not found"));
        
        if (livro.getQuantidadeDisponivel() < qty) {
            throw new RuntimeException(
                "Not enough books available");
        }
        
        livro.setQuantidadeDisponivel(
            livro.getQuantidadeDisponivel() - qty);
        livroRepository.save(livro);
    }
}
\end{lstlisting}
\end{frame}

\begin{frame}{Fluxo de Chamadas - Realizar Empréstimo}
\begin{center}
\begin{tikzpicture}[node distance=1.2cm]
    \node (frontend) [draw, rectangle, fill=blue!20] {Frontend: User clica "Emprestar"};
    \node (service) [draw, rectangle, fill=blue!20, below of=frontend] {Service: emprestimoService.create()};
    \node (controller) [draw, rectangle, fill=green!20, below of=service] {Controller: EmprestimoController};
    \node (businesslogic) [draw, rectangle, fill=green!20, below of=controller] {Service: Validações + Lógica};
    \node (repository) [draw, rectangle, fill=yellow!20, below of=businesslogic] {Repository: JPA save()};
    \node (database) [draw, rectangle, fill=red!20, below of=repository] {Database: SQL INSERT};
    
    \draw[->, thick] (frontend) -- (service);
    \draw[->, thick] (service) -- (controller) node[midway,right] {\tiny HTTP POST + JWT};
    \draw[->, thick] (controller) -- (businesslogic);
    \draw[->, thick] (businesslogic) -- (repository);
    \draw[->, thick] (repository) -- (database) node[midway,right] {\tiny JPA};
\end{tikzpicture}
\end{center}
\end{frame}

% ============================================================
% SEÇÃO 5: TESTES E QUALIDADE
% ============================================================
\section{Testes e Qualidade}

\begin{frame}{Estratégia de Testes}
\begin{block}{Testes Unitários}
\begin{itemize}
    \item Framework: JUnit 5 + Mockito
    \item Foco: Service Layer e Domain Logic
    \item Objetivo: Validar regras de negócio isoladamente
\end{itemize}
\end{block}

\begin{block}{Testes de Integração}
\begin{itemize}
    \item Framework: Spring Boot Test + MockMvc
    \item Foco: Fluxo completo Controller $\rightarrow$ Service $\rightarrow$ Repository
    \item Objetivo: Validar integração entre camadas
\end{itemize}
\end{block}

\begin{alertblock}{Cobertura}
Meta: > 70\% | Atual: \textasciitilde70\% (Service + Controller)
\end{alertblock}
\end{frame}

\begin{frame}[fragile]{Exemplo de Teste Unitário}
\begin{lstlisting}[language=Java]
@SpringBootTest
class LivroServiceTest {
    @Mock
    private LivroRepository livroRepository;
    
    @InjectMocks
    private LivroService livroService;
    
    @Test
    void testDecreaseQuantity_Success() {
        Livro livro = new Livro("Clean Code", "Martin", 
            "123", 2008, "PH", 10, 5);
        when(livroRepository.findById(1L))
            .thenReturn(Optional.of(livro));
        
        livroService.decreaseAvailableQuantity(1L, 2);
        
        assertEquals(3, livro.getQuantidadeDisponivel());
        verify(livroRepository).save(livro);
    }
}
\end{lstlisting}
\end{frame}

\begin{frame}{Ferramentas de Qualidade}
\begin{columns}
\column{0.5\textwidth}
\textbf{Backend}
\begin{itemize}
    \item \faCheckCircle[regular] JUnit 5
    \item \faCheckCircle[regular] Mockito
    \item \faCheckCircle[regular] Spring Boot Test
    \item \faCheckCircle[regular] Maven Surefire
\end{itemize}

\column{0.5\textwidth}
\textbf{Frontend}
\begin{itemize}
    \item \faCheckCircle[regular] ESLint
    \item \faCheckCircle[regular] Prettier
    \item \faCheckCircle[regular] Vite (build otimizado)
\end{itemize}
\end{columns}

\vspace{0.5cm}

\begin{block}{Boas Práticas Aplicadas}
\begin{itemize}
    \item Princípios SOLID
    \item Clean Code
    \item Git Workflow (commits descritivos, branches)
    \item Code Review via Pull Requests
\end{itemize}
\end{block}
\end{frame}

% ============================================================
% SEÇÃO 6: DEMONSTRAÇÃO
% ============================================================
\section{Demonstração}

\begin{frame}{Demonstração Funcional}
\begin{center}
\LARGE \faVideo\ Vídeo de Demonstração
\end{center}

\vspace{0.5cm}

\textbf{Fluxos Demonstrados:}
\begin{enumerate}
    \item Autenticação com JWT
    \item Gestão de Livros (CRUD)
    \item Registro de Empréstimo
    \item Sistema de Reservas
    \item Relatórios Administrativos
\end{enumerate}

\vspace{0.5cm}

\begin{center}
\faGithub\ \texttt{github.com/andreeluis/libshow}\\
\faYoutube\ \textit{[Link do vídeo será adicionado]}
\end{center}
\end{frame}

\begin{frame}{Funcionalidades Demonstradas}
\begin{columns}
\column{0.5\textwidth}
\textbf{Alunos}
\begin{itemize}
    \item \faCheckCircle[regular] Consultar acervo
    \item \faCheckCircle[regular] Visualizar histórico
    \item \faCheckCircle[regular] Fazer reservas
\end{itemize}

\textbf{Bibliotecários}
\begin{itemize}
    \item \faCheckCircle[regular] Registrar empréstimos
    \item \faCheckCircle[regular] Gerenciar livros
    \item \faCheckCircle[regular] Processar devoluções
\end{itemize}

\column{0.5\textwidth}
\textbf{Administradores}
\begin{itemize}
    \item \faCheckCircle[regular] Relatórios completos
    \item \faCheckCircle[regular] Estatísticas do sistema
    \item \faCheckCircle[regular] Livros mais emprestados
\end{itemize}

\textbf{Extras Implementados}
\begin{itemize}
    \item \faStar\ UI moderna (shadcn/ui)
    \item \faStar\ Interface responsiva
    \item \faStar\ Validações em tempo real
\end{itemize}
\end{columns}
\end{frame}

% ============================================================
% SEÇÃO 7: CONCLUSÕES
% ============================================================
\section{Conclusões}

\begin{frame}{Principais Aprendizados}
\begin{columns}
\column{0.5\textwidth}
\textbf{Técnicos}
\begin{itemize}
    \item Arquitetura em Camadas
    \item Spring Boot Ecosystem
    \item React Moderno (Hooks)
    \item APIs RESTful
    \item Autenticação JWT
    \item Testes Automatizados
\end{itemize}

\column{0.5\textwidth}
\textbf{Engenharia de Software}
\begin{itemize}
    \item Modelagem UML
    \item User Stories
    \item Padrões de Design
    \item Git Workflow
    \item Documentação
    \item Trabalho em Equipe
\end{itemize}
\end{columns}
\end{frame}

\begin{frame}{Desafios Enfrentados}
\begin{enumerate}
    \item \textbf{Integração Frontend-Backend}
    \begin{itemize}
        \item \faArrowRight\ Solução: Contrato de API bem definido
    \end{itemize}
    
    \item \textbf{Segurança com JWT}
    \begin{itemize}
        \item \faArrowRight\ Solução: Estudo aprofundado de Spring Security
    \end{itemize}
    
    \item \textbf{Gestão de Estado no React}
    \begin{itemize}
        \item \faArrowRight\ Solução: React Hooks (useState, useEffect)
    \end{itemize}
    
    \item \textbf{Testes de Integração}
    \begin{itemize}
        \item \faArrowRight\ Solução: MockMvc e contexto de testes
    \end{itemize}
\end{enumerate}
\end{frame}

\begin{frame}{Melhorias Futuras}
\begin{columns}
\column{0.5\textwidth}
\textbf{Curto Prazo}
\begin{itemize}
    \item Notificações por email
    \item Sistema de multas
    \item Upload de capas
    \item Histórico detalhado
    \item Filtros avançados
\end{itemize}

\column{0.5\textwidth}
\textbf{Médio/Longo Prazo}
\begin{itemize}
    \item Mobile App (React Native)
    \item Leitor de código de barras
    \item Dashboard Analytics
    \item API GraphQL
    \item Microservices
    \item CI/CD Pipeline
\end{itemize}
\end{columns}
\end{frame}

\begin{frame}{O Que Faríamos Diferente}
\begin{block}{Planejamento}
\begin{itemize}
    \item Definir API Contract antes (Swagger/OpenAPI)
    \item Modelagem UML mais detalhada no início
    \item Setup de CI/CD desde o começo
\end{itemize}
\end{block}

\begin{block}{Implementação}
\begin{itemize}
    \item DTOs separados das Entities
    \item Bean Validation (\texttt{@Valid}, \texttt{@NotNull})
    \item Exception Handling Global (\texttt{@ControllerAdvice})
    \item Logs estruturados (SLF4J)
\end{itemize}
\end{block}

\begin{block}{Testes}
\begin{itemize}
    \item TDD desde o início
    \item Meta de 90\%+ cobertura
    \item Testes E2E (Selenium/Cypress)
\end{itemize}
\end{block}
\end{frame}

% ============================================================
% SLIDE FINAL: RESUMO EXECUTIVO
% ============================================================
\begin{frame}{Resumo Executivo}
\begin{table}
\tiny
\begin{tabular}{|l|p{7cm}|}
\hline
\textbf{Aspecto} & \textbf{Detalhes} \\
\hline
Problema & Gestão manual de bibliotecas é ineficiente \\
\hline
Solução & Sistema web integrado e moderno \\
\hline
Arquitetura & 3 Camadas (React + Spring Boot + H2) \\
\hline
Padrões & MVC, Repository, DI, JWT \\
\hline
Tecnologias & Java 21, Spring Boot 3.3, React 19, JPA \\
\hline
Testes & JUnit 5, Mockito, Spring Test \\
\hline
Resultados & Sistema funcional e escalável \\
\hline
Aprendizados & Arquitetura, padrões, trabalho em equipe \\
\hline
\end{tabular}
\end{table}
\end{frame}

% ============================================================
% SLIDE FINAL: AGRADECIMENTOS
% ============================================================
\begin{frame}
\begin{center}
{\Huge Obrigado!}

\vspace{1cm}

{\Large \faBook\ LibShow}\\
\vspace{0.3cm}
{\large Sistema de Gerenciamento de Biblioteca Acadêmica}

\vspace{1cm}

\begin{tabular}{cl}
\faGithub & \texttt{github.com/andreeluis/libshow} \\
\faYoutube & \textit{[Link do vídeo de demonstração]} \\
\faEnvelope & \textit{[emails dos integrantes]} \\
\end{tabular}

\end{center}
\end{frame}

\end{document}
